\chapter{总结与展望}
\label{cha:summary}



\section{主要结论}
网络路由协议在网络信息传输的过程中起着至关重要的作用。iBGP路由全连接引起的路由可扩展问题一直是学术界关注的问题。目前工业界在比较大型的自治系统内均使用路由反射来解决iBGP路由的可扩展问题,但路由反射在特定情况下会产生非最优路由、转发环路、路由震荡等问题,如何设计一个合适的路由反射,当网络拓扑发生改变时路由反射结构该如何变化,这些也是路由反射潜在的问题。学术界除了路由反射,还提出过AS联邦、softRouter、RCP、RFCP等方案,这5种已有研究解决了iBGP的可扩展问题,但是仍存在不同程度的缺陷和局限性。

为了更好地解决iBGP存在的可扩展问题,本文提出了集中式路由体系结构下的RSCP-iBGP系统,该系统不仅解决了iBGP的可扩展问题,还优化了路由存储和路由计算,路由策略也能够在集中平台上很方便地进行配置。

本文主要完成了以下几个方面的工作:
\begin{itemize}
  \item 对BGP协议及其工作流程、iBGP协议存在的问题、现有的解决iBGP可扩展问题的相关研究、集中式思路优化BGP的相关研究进行了综述;
  \item 提出了基于集中平台的iBGP系统结构,介绍了系统架构、重要模块,将该系统与已有研究进行对比通过理论分析验证该系统的设计价值;
  \item 对虚拟路由器Quagga进行了介绍,设计并实现了RSCP-iBGP系统;
  \item 对RSCP-iBGP系统进行功能验证实验、性能评价试验以及一致性测试。
\end{itemize}


\section{未来工作}

受时间的限制,本文仅对路由存储和路由计算进行了优化,并没有充分利用集中控制平台的优势和资源,对路由策略、边界路由器的eBGP连接拓扑进行改进,未来还可以继续开展的工作主要有三处:

\begin{itemize}
  \item 集中配置路由策略后,实现策略正确性、覆盖性的自动化检测;
  \item 通过增加路由策略配置规则,使得路由策略配置更加灵活;
  \item 集中平台能够获取域内边界路由器与周边自治系统连接的拓扑信息,应该利用该信息优化边界路由器的eBGP连接,优化网络性能。
\end{itemize}