\chapter{总结与展望}
\label{cha:summary}



\section{主要结论}
网络路由协议在网络信息传输的过程中起着至关重要的作用。iBGP路由全连接引起的路由可扩展问题一直是学术界关注的问题。目前工业界在比较大型的自治系统内均使用路由反射来解决iBGP路由的可扩展问题,但路由反射在特定情况下会产生非最优路由、转发环路、路由震荡等问题,如何设计一个合适的路由反射,当网络拓扑发生改变时路由反射结构该如何变化,这些也是路由反射潜在的问题。学术界除了路由反射,还提出过AS联邦、SoftRouter、RCP、RFCP等方案,这5种已有研究解决了iBGP的可扩展问题,但是仍存在不同程度的缺陷和局限性。

为了更好地解决iBGP存在的可扩展问题,本文提出了集中式路由体系结构下的RSCP-iBGP系统,该系统不仅解决了iBGP的可扩展问题,还优化了路由存储和路由计算,消除了路由收敛和路由震荡。

本文主要完成了以下几个方面的工作:
\begin{itemize}
  \item 对BGP协议及其工作流程、iBGP协议存在的问题、现有的解决iBGP可扩展问题的相关研究、集中式思路优化BGP的相关研究进行了综述;
  \item 提出了基于集中平台的iBGP系统结构RSCP-iBGP,介绍了系统的架构和细节,将该系统与已有研究进行对比通过理论分析验证该系统的设计价值;
  \item 对虚拟路由器Quagga的基本概念和系统结构、BGP协议的路由存储、BGP协议的路由更新流程以及BGP协议的路由算法进行了介绍,设计并实现了RSCP-iBGP系统,具体介绍了RSCP-iBGP系统下三大部件边界路由器、路由集中控制平台、通信接口的设计与实现;
  \item 对RSCP-iBGP系统进行路由存储、路由计算、基于全部路由且无MED值引起的路由震荡等实验,从而验证RSCP-iBGP系统功能的正确性。此外,对RSCP-iBGP系统的Route-Server路由器路由宣告、Route-Client路由器的路由宣告以及UPDATE报文等进行了一致性测试,验证了RSCP-iBGP系统功能与设计的一致性。
\end{itemize}


\section{未来工作}

受时间的限制,本文仅对路由存储和路由计算进行了优化,并没有充分利用集中控制平台的优势和资源,对路由策略、边界路由器的eBGP连接拓扑进行改进和优化,另外本文未对RSCP-iBGP系统进行性能评价实验,未来还可以继续开展的工作主要有三处:

\begin{itemize}
  \item 针对本文提出的RSCP-iBGP系统,可增加系统性能评价实验,如大规模路由信息进入自治系统时路由存储所占用的空间大小、路由收敛时间评估等实验;
  \item 可以实现集中配置路由策略后,实现策略正确性、覆盖性的自动化检测,通过增加路由策略配置规则,使得路由策略配置更加灵活;
  \item 集中平台能够获取域内边界路由器与周边自治系统连接的拓扑信息,应该利用该信息优化边界路由器的eBGP连接,优化网络性能。
\end{itemize} 