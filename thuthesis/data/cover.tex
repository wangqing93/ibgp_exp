\thusetup{
  %******************************
  % 注意:
  %   1. 配置里面不要出现空行
  %   2. 不需要的配置信息可以删除
  %******************************
  %
  %=====
  % 秘级
  %=====
  secretlevel={公开},
  secretyear={2100},
  %
  %=========
  % 中文信息
  %=========
  ctitle={基于集中平台的新型iBGP系统的研究与实现},
  %清华大学学位论文 \LaTeX\ 模板\\使用示例文档 v\version},
  cdegree={工学硕士},
  cdepartment={计算机科学与技术系},
  cmajor={计算机科学与技术},
  cauthor={王庆},
  csupervisor={尹霞教授},
  %cassosupervisor={陈文光教授}, % 副指导老师
  %ccosupervisor={某某某教授}, % 联合指导老师
  % 日期自动使用当前时间,若需指定按如下方式修改:
   cdate={二〇一八年五月},
  %
  % 博士后专有部分
  cfirstdiscipline={计算机科学与技术},
  cseconddiscipline={系统结构},
  postdoctordate={2009年7月——2011年7月},
  id={编号}, % 可以留空: id={},
  udc={UDC}, % 可以留空
  catalognumber={分类号}, % 可以留空
  %
  %=========
  % 英文信息
  %=========
  etitle={Research and Implementation of Advanced iBGP System Based on Centralized Platform},
  % 这块比较复杂,需要分情况讨论:
  % 1. 学术型硕士
  %    edegree:必须为Master of Arts或Master of Science(注意大小写)
  %             “哲学、文学、历史学、法学、教育学、艺术学门类,公共管理学科
  %              填写Master of Arts,其它填写Master of Science”
  %    emajor:“获得一级学科授权的学科填写一级学科名称,其它填写二级学科名称”
  % 2. 专业型硕士
  %    edegree:“填写专业学位英文名称全称”
  %    emajor:“工程硕士填写工程领域,其它专业学位不填写此项”
  % 3. 学术型博士
  %    edegree:Doctor of Philosophy(注意大小写)
  %    emajor:“获得一级学科授权的学科填写一级学科名称,其它填写二级学科名称”
  % 4. 专业型博士
  %    edegree:“填写专业学位英文名称全称”
  %    emajor:不填写此项
  edegree={Master of Science},
  emajor={Computer Science and Technology},
  eauthor={Wang Qing},
  esupervisor={Professor Yin Xia},
  %eassosupervisor={Chen Wenguang},
  % 日期自动生成,若需指定按如下方式修改:
   edate={May, 2018},
  %
  % 关键词用“英文逗号”分割
   ckeywords={内部边界网关协议, 集中平台, 路由存储, 路由策略, 路由计算},
   ekeywords={Internal Border Gateway Protocol, Centralized Platform, Routing Storage, Routing Policy, Routing Calculation}
}


% 定义中英文摘要和关键字
\begin{cabstract}
  随着网络规模的不断扩大,且内部边界网关协议iBGP要求自治系统内部的边界路由器全连接来传播路由信息,iBGP协议可扩展性问题日益显著。已有的研究方案路由反射、AS联邦、SoftRouter、RCP、RFCP等解决了iBGP的可扩展性,但带来了新的挑战,如路由震荡、转发环路、路由表冗余存储等。本文基于集中式体系结构的思路,提出基于集中平台的RSCP-iBGP系统,在解决iBGP的可扩展问题的同时,优化了路由存储和路由计算。部署RSCP-iBGP系统的自治系统,其自治系统内的路由存储空间和路由计算次数均下降了一个数量级。通过在特定拓扑上部署RSCP-iBGP系统对其进行功能评价试验,以及使用协议一致性测试工具PITSv3对RSCP-iBGP系统的一致性进行测试和验证,最终验证RSCP-iBGP系统功能实现与设计的一致性。

  本文主要开展了以下工作:
  \begin{itemize}
    \item 针对iBGP可扩展问题,提出了基于集中平台的RSCP-iBGP系统,其中对路由存储、路由计算进行了优化,并将该系统与已有的5种方案进行对比分析,明确RSCP-iBGP系统相比现有研究方案的优势。
    \item 设计RSCP-iBGP系统的工作流程和具体模块,实现RSCP-iBGP系统的主要组成部分:集中平台上的Route-Server,边界路由器Route-Client以及边界路由器与集中平台之间的通信协议。集中平台上的Loc-RIB增量存储和复式路由计算模块,使得自治系统内的路由存储空间和路由计算次数均下降了一个数量级。
    \item 通过在特定拓扑上运行RSCP-iBGP系统,对该系统进行功能验证实验。通过一致性测试工具PITSv3,编写测试例对RSCP-iBGP系统进行功能一致性测试。最终验证了RSCP-iBGP系统的功能和设计相一致。
  \end{itemize}

\end{cabstract}

% 如果习惯关键字跟在摘要文字后面,可以用直接命令来设置,如下:
% \ckeywords{\TeX, \LaTeX, CJK, 模板, 论文}

\begin{eabstract}
   With the continuous expansion of network scale, internal border gateway protocol iBGP which requires a logical full mesh occurs scalability issue. Previous studies, including Route Reflection, AS Confederation, SoftRouter, RCP, RFCP, have solved the iBGP scalability issue, but bring new challenges, such as routing oscillation, forwarding loop, routing table redundancy storage, etc. This thesis proposes RSCP-iBGP system based on centralized platform to solve the iBGP scalability issue, which also optimizes routing storage and routing calculation. The RSCP-iBGP system based on centralized platform contributes that the size of routing storage and the number of routing computation are both reduced by an order of magnitude. With executing functional verification of RSCP-iBGP system by specific topology and conformance testing by the conformance testing tool PITSv3, this thesis finally verified that the realization of the RSCP-iBGP system is consistent with the design. The contributions of the work are as follows:

   \begin{itemize}
    \item Put forward the RSCP-iBGP system based on centralized platform, which solves scalability issue and optimizes routing storage and routing calculation. By comparing RSCP-iBGP system with existing five schemes, the advantages of RSCP-iBGP system are confirmed.
    \item Design the working process and specific modules in RSCP-iBGP system, and realize major components: Route-Client as the border router, Route-Server running in the centralized platform and the communication protocol.
    \item Execute functional verification of RSCP-iBGP system by specific topology. The conformance testing of the RSCP-iBGP system is conducted by the conformance testing tool PITSv3. These experiments finally have verified that the realization of the RSCP-iBGP system is consistent with the design.
  \end{itemize}
\end{eabstract}




% \ekeywords{\TeX, \LaTeX, CJK, template, thesis}
