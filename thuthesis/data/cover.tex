\thusetup{
  %******************************
  % 注意:
  %   1. 配置里面不要出现空行
  %   2. 不需要的配置信息可以删除
  %******************************
  %
  %=====
  % 秘级
  %=====
  secretlevel={公开},
  secretyear={2100},
  %
  %=========
  % 中文信息
  %=========
  ctitle={基于集中平台的iBGP系统的研究与实现},
  %清华大学学位论文 \LaTeX\ 模板\\使用示例文档 v\version},
  cdegree={工学硕士},
  cdepartment={计算机科学与技术系},
  cmajor={计算机科学与技术},
  cauthor={王庆},
  csupervisor={尹霞教授},
  %cassosupervisor={陈文光教授}, % 副指导老师
  %ccosupervisor={某某某教授}, % 联合指导老师
  % 日期自动使用当前时间,若需指定按如下方式修改:
  % cdate={超新星纪元},
  %
  % 博士后专有部分
  cfirstdiscipline={计算机科学与技术},
  cseconddiscipline={系统结构},
  postdoctordate={2009年7月——2011年7月},
  id={编号}, % 可以留空: id={},
  udc={UDC}, % 可以留空
  catalognumber={分类号}, % 可以留空
  %
  %=========
  % 英文信息
  %=========
  etitle={A Study and Implement Based on Centralized Platform of IBGP System},
  % 这块比较复杂,需要分情况讨论:
  % 1. 学术型硕士
  %    edegree:必须为Master of Arts或Master of Science(注意大小写)
  %             “哲学、文学、历史学、法学、教育学、艺术学门类,公共管理学科
  %              填写Master of Arts,其它填写Master of Science”
  %    emajor:“获得一级学科授权的学科填写一级学科名称,其它填写二级学科名称”
  % 2. 专业型硕士
  %    edegree:“填写专业学位英文名称全称”
  %    emajor:“工程硕士填写工程领域,其它专业学位不填写此项”
  % 3. 学术型博士
  %    edegree:Doctor of Philosophy(注意大小写)
  %    emajor:“获得一级学科授权的学科填写一级学科名称,其它填写二级学科名称”
  % 4. 专业型博士
  %    edegree:“填写专业学位英文名称全称”
  %    emajor:不填写此项
  edegree={Master of Science},
  emajor={Computer Science and Technology},
  eauthor={Wang Qing},
  esupervisor={Professor Yin Xia},
  %eassosupervisor={Chen Wenguang},
  % 日期自动生成,若需指定按如下方式修改:
  % edate={December, 2005}
  %
  % 关键词用“英文逗号”分割
   ckeywords={内部边界网关协议, 集中平台, 路由存储, 路由策略, 路由计算},
   ekeywords={Internal Border Gateway Protocol, Centralized Platform, Routing Storage, Routing Policy, Routing Calculation}
}


% 定义中英文摘要和关键字
\begin{cabstract}
  随着网络规模的不断扩大,自治系统内部需要逻辑全连接的内部边界网关协议iBGP,面临可扩展性问题。已有的研究方案路由反射、AS联邦、SoftRouter、RCP、RFCP等解决了iBGP的可扩展性,但带来了新的挑战。本文基于集中式体系结构的思路,提出基于集中平台的iBGP扩展协议,并基于集中平台的集中式优势,在解决iBGP的可扩展问题的同时,也优化了路由存储和路由计算。如果自治系统内有N台边界路由器,iBGP全连接需要维护N平方的iBGP连接,路由表存储需要N份,路由计算需要最坏N平方次,而文中的新方案仅需维护N的iBGP连接,路由表存储1份,路由表最坏计算N次,基于集中平台的iBGP协议使得内部域间路由协议iBGP的性能、可扩展性提升很多。

  本文主要开展了以下工作:
  \begin{itemize}
    \item 设计并实现基于集中平台的iBGP扩展路由协议。
    \item 优化路由存储和路由计算。
    \item 使用TTCN-3编写测试例对iBGP扩展协议进行功能一致性测试。
  \end{itemize}

\end{cabstract}

% 如果习惯关键字跟在摘要文字后面,可以用直接命令来设置,如下:
% \ckeywords{\TeX, \LaTeX, CJK, 模板, 论文}

\begin{eabstract}
   With the development of network, internal border gateway protocol iBGP which requires a logical full mesh occurs scalability, security issues. Existing research solved the iBGP scalability issue, but brought new challenges,such as route reflection, AS federal, SoftRouter, RCP, RFCP...etc. We propose iBGP extended protocol based on centralized platform to solve the iBGP scalability issue, which also optimized routing storage and routing calculation. If there are N border routers in one Autonomous system, IBGP needs to maintain N squared iBGP connections, to store N routing tables , and routing computing needs to execute N squared times. Our iBGP extended protocol only needs to maintain N iBGP connections, one storage routing table, and routing computing needs to execute N times. The iBGP extended protocol based on centralized platform improves the internal inter-domain routing protocols iBGP performance and scalability. The contributions of the work include:

   \begin{itemize}
    \item Design and implement an iBGP extended routing protocol based on centralized platform.
    \item Optimize routing storage and routing calculations.
    \item Realize functional consistency testing of the iBGP extension protocol by TTCN-3 test cases.
  \end{itemize}



\end{eabstract}




% \ekeywords{\TeX, \LaTeX, CJK, template, thesis}
