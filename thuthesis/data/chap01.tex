\chapter{引言}
\label{cha:intro}


\section{研究背景}

随着互联网的不断发展,自治系统的规模不断扩大,自治系统内的边界路由器数目和自治系统向外宣告前缀数目不断增加,互联网的域间路由协议面临多方面的挑战。边界网关协议(Border Gateway Protocol,BGP)是目前互联网使用的唯一的域间路由协议,直接影响网络功能和端到端的传输。早期设计BGP协议的阶段,自治系统(Autonomous System,AS)的数量和自治系统内部的边界路由器数量均较小,对BGP协议的可扩展性、安全性、路由策略的灵活性等缺乏关注。

BGP在AS内部所有路由器两两之间建立iBGP(Internal Border Gateway Protocol, iBGP)会话以交换域间路由,iBGP会话的数量是路由器平方数量级的,大型AS内部边界路由器的数目几百到上千,难以承受巨大的iBGP负担。学术界2000年左右提出了路由反射(Router Reflector, RR)和AS联邦的解决方案,两种方案虽然解决了全连接引起的可扩展性差的问题,但因为部分边界路由器缺少路由信息,带来了新的问题,比如非最优出口路由、转发环路、路由震荡等等,没有很好的解决当前网络环境下,iBGP面临的挑战。之后,有学者提出了基于集中式体系结构的SoftRouter、RCP(Route Control Platform)、RFCP(Route Flow Control Platform)等研究方案,在解决iBGP的可扩展问题的同时,保证所有边界路由器的路由决策基于全部的路由信息,解决了路由反射和AS联邦引发的新问题,同时该体系结构为路由存储、策略管理、路由计算提供了集中处理和管理的平台,使得内部域间路由具有很大的优化空间。

现有的基于集中式体系结构的iBGP研究,对路由存储、策略管理、路由计算的某些方面进行了优化,本文希望综合已有研究,在解决iBGP可扩展性的同时,将iBGP的路由存储、策略管理、路由计算的优化达到最大。

\section{主要工作}

本文对基于集中平台的RSCP-iBGP系统进行了实现、测试和评估,主要进行了以下工作:
\begin{enumerate}
\item 对BGP协议及其工作流程、域间路由协议iBGP存在的问题、以及解决iBGP存在问题的RR、AS联邦、SofterRouter、RCP、RFCP等研究方案、集中式思路优化BGP的相关研究进行了综述;
\item 介绍基于集中平台的RSCP-iBGP系统的整体框架,重点介绍RSCP-iBGP系统中集中平台、边界路由器、扩展的iBGP路由协议三大部件,并将该系统与现有的5种研究方案进行对比分析;
\item 对虚拟路由器Quagga的系统结构及路由更新等细节进行介绍,设计并实现基于集中平台的RSCP-iBGP系统;
\item 对RSCP-iBGP系统进行功能验证实验,对RSCP-iBGP系统进行协议的一致性测试,通过验证RSCP-iBGP系统的功能实现性和一致性。
\end{enumerate}



\section{论文结构}
本文共包含六章:
\begin{itemize}
\item 第一章:引言,介绍研究背景和主要工作;
\item 第二章:相关研究综述,对BGP协议及工作流程、iBGP存在问题、iBGP可扩展性问题的研究现状、集中式思路优化BGP的相关研究等进行了综述;
\item 第三章:描述了基于集中平台的RSCP-iBGP系统的整体框架、系统部件,将其与已有方案进行对比分析;
\item 第四章:介绍软件路由器Quagga中BGP协议的实现细节,设计并实现基于集中平台的RSCP-iBGP系统,包括边界路由器、路由控制平台以及通信接口;
\item 第五章:对基于集中平台的RSCP-iBGP系统进行仿真实验和一致性测试,设计实验拓扑和一致性测试集,在实验床上运行扩展型iBGP协议,验证RSCP-iBGP系统的功能实现性和一致性;
\item 第六章:总结和进一步研究工作。
\end{itemize}
