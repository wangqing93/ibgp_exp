\begin{resume}

  \resumeitem{个人简历}

  1993年5月4日出生于陕西省韩城县。

  2011年8月考入清华大学计算机科学与技术系,2015 年7月本科毕业并获得工学学士学位。

  2015年9月免试进入清华大学计算机科学与技术系攻读硕士学位,导师尹霞教授。

  \researchitem{发表的学术论文} % 发表的和录用的合在一起

  % 1. 已经刊载的学术论文(本人是第一作者,或者导师为第一作者本人是第二作者)
  \begin{publications}
    \item 王庆,王之梁,姚姜源,施新刚,尹霞,等.天地一体化网络新型路由协议一致性测试研究[J]. 中国电子科学研究院学报,2018(01):72-80.
  \end{publications}

  % 2. 尚未刊载,但已经接到正式录用函的学术论文(本人为第一作者,或者
  %    导师为第一作者本人是第二作者)。
  %\begin{publications}[before=\publicationskip,after=\publicationskip]
%    \item Yang Y, Ren T L, Zhu Y P, et al. PMUTs for handwriting recognition. In
%      press. (已被 Integrated Ferroelectrics 录用. SCI 源刊.)
%  \end{publications}
%
%  % 3. 其他学术论文。可列出除上述两种情况以外的其他学术论文,但必须是
%  %    已经刊载或者收到正式录用函的论文。
%  \begin{publications}
%    \item Wu X M, Yang Y, Cai J, et al. Measurements of ferroelectric MEMS
%      microphones. Integrated Ferroelectrics, 2005, 69:417-429. (SCI 收录, 检索号
%      :896KM)
%    \item 贾泽, 杨轶, 陈兢, 等. 用于压电和电容微麦克风的体硅腐蚀相关研究. 压电与声
%      光, 2006, 28(1):117-119. (EI 收录, 检索号:06129773469)
%    \item 伍晓明, 杨轶, 张宁欣, 等. 基于MEMS技术的集成铁电硅微麦克风. 中国集成电路,
%      2003, 53:59-61.
%  \end{publications}

  \researchitem{参与的科研项目} % 有就写,没有就删除
  \begin{achievements}
    \item 中国电子科学研究院合作项目. 天地一体化信息网络概念演示系统。
    \item IPv6大规模编址与路由关键技术研究和验证项目。
    \item 发改委下一代互联网技术研发、产业化和规模商用专项”网络及网站IPv6支持度评测体系及平台建设项目。
  \end{achievements}

\end{resume}
