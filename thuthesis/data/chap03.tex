\chapter{基于集中平台的iBGP系统结构}
\label{cha:china}


\section{本章引言}
可扩展问题是互联网iBGP协议的重要问题,随着网络规模和需求的不断增加,大多数自治系统已经不采用传统的Full-mesh的iBGP连接结构,而是使用配置比较方便、技术比较成熟的路由反射。现有的解决iBGP可扩展问题的思路可分为两种:分布式路由体系结构、集中式路由体系结构。分布式路由体系结构下的相关研究主要有路由反射和AS联邦,其解决了iBGP的可扩展问题,但是带来了新的问题,比如:非最优出口、转发环路、路由震荡。而集中式路由体系结构的相关研究主要有SoftRouter、RCP、RFCP三种代表性方案,这三种方案解决了iBGP的可扩展问题,但也分别存在集中平台本身的可扩展性差、路由存储冗余重复、未优化的传统路由计算方式等等问题。

本文基于数据平面和控制平面分离的思想,提出了基于集中平台的iBGP系统结构,将自治系统内部的路由存储、路由计算以及路由转发交换功能合理划分为集中平台和自治系统内边界路由器上执行的两部分。该基于集中平台的iBGP系统结构处理路由信息的基本流程为:自治系统内的边界路由器收到路由信息,通过扩展的iBGP协议将经过入站策略的路由信息发送到集中平台,集中平台为每台自治系统内的边界路由器计算出最优路由,再次通过扩展的iBGP协议将最优路由传输给域内的每台边界路由器,边界路由器将经过出站策略的路由信息宣告出去。此外,集中平台的存在,为路由存储、路由计算的优化提供了很多可能,本文提出的基于集中平台的iBGP系统,不仅解决了iBGP存在的可扩展问题,而且对自治系统内BGP的路由存储和计算进行了优化。

本章首先提出基于集中平台的iBGP系统架构,之后对系统中的三个主要模块扩展iBGP协议、增量路由存储、复式路由计算进行详细的介绍,最后主要从解决iBGP可扩展问题、与分布式路由体系结构方案的对比、与集中式路由体系结构方案的这三个方面对基于集中平台的iBGP系统进行定性的分析。
\section{iBGP路由系统架构}
总体结构,系统工作流程图


\section{系统模块介绍}
\subsection{扩展iBGP协议}
\subsection{增量路由存储}
\subsection{复式路由计算}
\section{理论分析}
\subsection{解决iBGP可扩展问题}
\subsection{与分布式路由体系结构方案的对比}
\subsection{与集中式路由体系结构方案的对比}
\subsection{总结}

\section{本章小结}
总结