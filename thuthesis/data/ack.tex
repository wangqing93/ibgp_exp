% 如果使用声明扫描页,将可选参数指定为扫描后的 PDF 文件名,例如:
% \begin{acknowledgement}[scan-statement.pdf]
\begin{acknowledgement}
  衷心感谢导师尹霞教授,在我读硕士期间对我的关心和指导,并言传身教引导我树立严谨的科研态度和认真踏实的做事原则。在我科研长时间停滞不前的阶段,是尹霞老师的鼓励和支持,帮助我重拾科研的信心,语重心长地告诫我在科研上投入的时间不足,应“先做减法,再做加法”,让我迅速走出了科研困境。尹霞老师对实验室的同学都非常爱护,切身考虑同学们的处境和需要,关键时刻给我们很多精神上的大力支持和切合实际的宝贵建议,特别是在我毕业找工作的过程中,给予了我很多的帮助和建议。她不仅是我科研上的引领者,更是我人生的导师。

  万分感谢我的辅导老师王之梁老师,他批判性的科研思维和细致严谨的科研习惯,时时刻刻影响着我。硕士期间王老师一直悉心指导我的科研,定期与我讨论,明确指出我在科研上的一些陋习和思维缺陷,帮助我在科研上不断成长和进步。我逻辑思维的缜密度、对学术研究的对比分析和总结归纳能力等在王老师的帮助下有显著提升。王老师在科研上对我的指导非常细致,有时我在实验阶段遇到瓶颈,王老师会亲自帮我查看并调试,我也能从中学习老师解决实际问题的方法,非常感激老师。以后在学习新东西的过程中,我也将继续保持在王老师教诲下养成的良好思维习惯及学习方式。

  非常感谢施新刚老师、姚姜源老师,他们对我在实验室的学习给了很大的帮助和支持。施老师一丝不苟的科研态度以及广泛的涉猎并深入学习数学、算法、计算机网络等领域的习惯,也不断激励着我。姚老师精通协议的一致性测试,在我实践协议一致性测试以及论文发表的过程中,姚老师耐心帮我讲解一致性测试工具的原理和使用注意事项,认真帮我修改论文中的部分段落,让我在协议的一致性测试以及论文发表的过程中少走了很多弯路,真诚地感谢姚老师!

  感谢实验室余家傲老师、张娜老师对我生活上的帮助,为实验室的同学创造了良好的科研环境。感谢实验室的张晗、郭迎亚、赵宗义、李亚慧、杨言、王苏、刘智峰、陈志鹏、马强、田莹等同学的帮助和鼓励,和他们的讨论帮助我解决了很多科研上的盲点,他们对我的鼓励促使我更加用心地投入到科研,感谢你们!


  %感谢 \LaTeX 和 \thuthesis\cite{thuthesis},帮我节省了不少时间。
\end{acknowledgement}
